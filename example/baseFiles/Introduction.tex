\section{Wstęp}
Wyznaczanie stałej Plancka na podstawie charakterystyki diody elektroluminescencyjnej”, opiera się na stopniowym mierzeniu i zwiększaniu wartości napięcia oraz natężenia.
Dzięki właściwością diody możliwe jest wyznaczenie stałej Plancka, energia naszego fotonu będzie równa energii potrzebnej do przekroczenia bariery potencjału, czyli $E=eU_b$, jednak wiemy, że energia fotonu to $E=hf$, a $f=\frac{c}{\lambda}$ , z tego wynika, że $h=\frac{e}{c}\lambda U_b$, na podstawie tego wzoru oparliśmy nasze końcowe obliczenia i dzięki niemu, mogliśmy wyznaczyć stałą.
Doświadczenie zostało zbadane dla dwóch diod elektroluminescencyjnych, czerwonej i żółtej.